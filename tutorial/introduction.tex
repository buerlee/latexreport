\section{前言}

在动笔之前,我们冒昧地对读者做出了一个大胆的假设:\emph{您刚刚听
  说\LaTeX{},虽然所知不多,但已决意学习。}之所以需要做出这样的假设,纯粹
是因为我们想回避一个力所不及的问题:\emph{为什么要学习\LaTeX{}}。本文意在
简介,只是扮演一个门童的角色,为您登堂入室带个路,无意以偏盖全。倘若被某
位行家看到错误之处,还望指正。

\subsection{\LaTeX{}的历史}

在不了解\LaTeX{}的背景知识之前可以学习它吗?当然可以。但考虑到每个人的学
习习惯并不一样,这里提供一些链接,让好奇心更强的读者在学习之前大快朵颐一
番。

\begin{itemize}
\item 中文:\url{http://www.ctex.org/documents/shredder/tex_frame.html}
\item 英文:\url{https://en.wikipedia.org/wiki/LaTeX}
\end{itemize}

如果对这两个链接你没兴趣,这里有个较为简捷的历史回顾:

\emph{
  \TeX{}是1978年被
  \href{https://en.wikipedia.org/wiki/Donald_Knuth}{Donald Knuth}搞出来的
  排版系统。这套玩意好是好,但搞起来比较麻烦,所以后来
  被\href{https://en.wikipedia.org/wiki/Leslie_Lamport}{Leslie
    Lamport}(1994)改进了一下形成了\LaTeX{}。}

第一次就能读准它们很重要:\TeX{}——“泰赫”;\LaTeX{}——“雷泰赫”。之所以
称它们为系统,是因为它们在核心程序之外,还要辅之以字体生成程序、字体、模
板(\textit{.cls}文件)、宏包(\textit{.sty}文件)等。由于\LaTeX{}提供的语法
支撑了我们今天的绝大部分\textit{tex}文档,所以我们很少提及\TeX{}了(一般
称用原始\TeX{}命令写的文档为\texttt{plain tex})。

此后为了让系统生成现在普遍使用的\texttt{PDF}文件,人们又开始扩展系统形成
了新的编译程序,现在叫\emph{编译引擎}:pdfTeX和pdfLaTeX。接着还要解决非英
文字体的支持问题,就又产生了:XeTeX和XeLaTeX。最后有人还将Lua和\TeX{}混合
到一起,做成了更适合编程的LuaTeX。所有这些扩展放在一起统
称
\href{https://www.sharelatex.com/blog/2012/12/01/the-tex-family-tree-latex-pdftex-xelatex-luatex-context.html}{\emph{\TeX{}
    家族}}。

\subsection{自学资料}

如果你没兴趣看本文的解释,情愿自己去学习权威教程或获取更丰富的资源。那就
给你推荐几个好去处。

\begin{itemize}
\item \url{https://en.wikibooks.org/wiki/LaTeX}
\item \url{http://www.tug.org/twg/mactex/tutorials}
\item \LaTeX{} for Complete Novices: 本地文档,执行命令打开该文档。
  \begin{cmd} texdoc dickimaw-novices
  \end{cmd}
 \item \LaTeXe{}: An unofficial reference manual
  \begin{cmd} texdoc latex2e
 \end{cmd}
 \item The Not So Short Introduction to \LaTeXe{}
   \begin{cmd} texdoc lshort
   \end{cmd}
\item 刘海洋. \LaTeX{}入门,电子工业出版社,2013
\item TUG: \url{http://www.tug.org},\TeX{}用户群(\TeX Users Group,
  TUG)网站,全世界\TeX{}用户的组织。这里可以下载到\texttt{TeXLive CD}。
\item CTAN: \url{http://www.ctan.org},(Comprehensive TeX Archive
  Network,CTAN),你想要的所有文档、模板都在这里。再强调一下,是所有的都
  在这里。
\item LaTeX: \url{http://www.latex-project.org/},\LaTeX{}官方网站。
\item Stackexchange:\url{http://stackexchange.com},专业的技术交流网站。
\item \LaTeX{}工作室:\url{http://www.latexstudio.cn},提供国内稀有高水准排版
  服务。还有一个QQ社区:91940767。
\end{itemize}

\subsection{安装}

工欲善其事必先利其器,了解\LaTeX{}之前先说说\uline {安装}这个话
题。\LaTeX{}的器包括\uline{系统}和\uline{编辑器}两部分。

\uline{系统}是很多人的习惯说法,更准确的说法是\uline{宏集}或\uline{套装},
反正就是安装之后可以编译\textit{tex}文档的软件吧。系统主要有三种:
\begin{itemize}
\item \href{https://miktex.org/}{MikTeX}:用于\texttt{Windows}系统环境;
\item \href{https://www.tug.org/mactex/}{MacTeX}:用于\texttt{Mac OSX}系
  统环境;
\item \href{http://tug.org/}{TeXLive}:用
  于\texttt{Unix}、\texttt{Linux}、\texttt{Windows}系统环境。据说
  \texttt{TeXLive}所拥有的宏包是最新的和最全面的;
\item \href{http://www.ctex.org/HomePage}{CTeX}:用于\texttt{Windows}系
  统环境。CTeX是国人以\texttt{MikTeX}为核心打造的\LaTeX{}套装,用来支持
  中文和中文环境。
\end{itemize}

为嘛叫\uline{宏集}或\uline{套装}呢?因为一般只有最原始
的\TeX{}或\LaTeX{}程序、模板和宏包远远满足不了用户的需求,用户还需要更丰
富的宏包、模板、字体、特别还需要编辑器。因此我们的前辈就东拼西凑来这些东
东,成了我们所说的宏集或套装。

开始动手下载吧,以\texttt{TeXLive}为例,它来源于一个所
谓\texttt{CTAN}(Comprehensive \TeX{} Archive Network)的网站,它在全世界各
地都有镜像网站,我们可以从就近从镜像网站上下载到安装包。这里提供一个
吧:
\href{http://mirrors.ustc.edu.cn/CTAN/systems/texlive/Images/texlive2017.iso}{texlive2017}
。

\textit{tex}编辑器五花八门,有\texttt{TeXLive}自带
的TeXWorks、\texttt{CTeX}自带的WinEdt,还有TeXMaker、TeXStudio等。除了这
些专用编辑器,通用编译器配上一插件也可以担当重任,
如Vim、Emacs、Atom、VisualStudio等。具体选择哪一款全凭个人感觉了。

%%% Local Variables:
%%% mode: latex
%%% TeX-master: "latexwritting"
%%% End:
