\section{基本命令和环境}

\emph{命令}和\emph{环境}是构成文档的基本单元。

命令的基本基本结构是:

\begin{code}
  <命令名>[选项]{参数}
\end{code}

环境是由一对命令构成的:
\begin{code}
 \begin{环境名}[选项] 
   % 环境体
 \end{环境名}
\end{code}

下面我们将分别介绍一些基本的命令和环境。

\subsection{题目、作者和日期}

文章的开头首先是题目和作者、日期。这三个信息都以命令形式定义的。它们都需
要在导言区定义,然后在正文中用一条命令\verb!\maketitle!将它们显示出来。

\begin{code}
  ...
  \title{都闪开,我要用\LaTeX{}写作了}
  \author{西门吹牛}
  \date{2015年某月某日}
  ...
  \begin{document}
    \maketitle % 没有这条命令,前面的定义都烟消云散。...
  \end{document}
\end{code}

\subsection{中文输入}

在导言区只要引用了宏包\verb!ctex!,即可用命令\verb!xelatex!编译中文了。注
意,一般要在选项中说明字体编码为\verb!UTF8!。

\begin{code}
  \usepackage[UTF8]{ctex}
\end{code}

\subsection{段落}

分段有两种方式:
\begin{itemize}
\item 在段落后填加\verb!\\!
\item 在段落后填加一个空行。
\end{itemize}

\subsection{章节}

章节命令能用到的就三个:
\begin{itemize}
\item 章:\verb!\chapter{绪论}!,在我们这个模板中这一级题目是无效的。
\item 节:\verb!\section{前言}!
\item 次节:\verb!\subsection{背景}!
\end{itemize}

\subsection{枚举}

\begin{codeout}
  \begin{itemize}
  \item 首先,我要强调。
  \item 其次,我还要说明。
  \end{itemize}
\end{codeout}

\subsection{数学公式}

参考阅读:\url{https://en.wikibooks.org/wiki/LaTeX/Mathematics}

\begin{codeout}
  % \usepackage{amsmath}一行文字之间的公式叫行内公式,比如:
  $x^2=1$。行间公式用公式环境,比如:
  \begin{eqnarray}
    \int_{0}^{1} x^2 \mathrm{d}x & = & \dfrac{1}{3} \\
    \int_{0}^{1} x^2 \mathrm{d}x & = & \sqrt{ \dfrac{1}{3}} 
  \end{eqnarray}
  式中:注意$\mathrm{d}$是正体,表示微分符号。
\end{codeout}

\subsection{特殊字符}

在文档写作中,经常涉及希腊字母或其它特殊字符的输入。\LaTeX{}已经状备好了
这些字符,只需要通过命令实即可实现输入。

\begin{codeout}
  %\usepackage{amsmath}
  $\alpha$ $\beta$ $\xi$ $\sigma$ $\theta$
\end{codeout}

这些字符已经汇总到了一个文档中供查询—\texttt{symbols-a4.pdf}。在字符界面
中,输入如下命令即可自动找到该文档并用系统默认的pdf浏览器打开:

\begin{cmd}
  > texdoc symbols-a4
\end{cmd}

还有一些网站专门提供了手写识别\LaTeX{}符号的功能,只需要用鼠标画出查询的
符号,网站就会推荐出可能的符号命令。这里推荐一个:

\subsection{颜色}
\url{https://www.sharelatex.com/learn/Using_colours_in_LaTeX}

\begin{codeout}
  \textcolor{red}{红色}

  {\color{blue} \rule{\linewidth}{0.5mm}}

  \colorbox{BurntOrange}{颜色}
\end{codeout}

\subsection{参考文献}

命令\verb!\bibliography!用来引用某个\texttt{bib}文件,如:
\begin{code}
  \bibliography{./Reference/Reference.bib}
\end{code}

在\texttt{bib}文件中的所有文献都有一个\texttt{Bibtexkey}项代表该文献,在
\texttt{tex}文件中用\verb!\cite!命令即可通过该项值引用文献,如:

\begin{code}
  文献\cite{Wu2010}指出了如下观点:
\end{code}

\subsection{目录}

插入目录只需一条命令,位于\verb!\maketitle!之后:

\begin{code}
  \tableofcontents
\end{code}

\url{http://detexify.kirelabs.org/classify.html}
%%% Local Variables:
%%% mode: latex
%%% TeX-master: "latexwritting"
%%% End:
